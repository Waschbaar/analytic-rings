\documentclass{article}

\usepackage{fontspec}
\usepackage{amsmath}
\usepackage{amsthm}
\usepackage{amssymb}
\usepackage{amsfonts}
\usepackage{mathrsfs}
\usepackage{tikz}
\usepackage{tikz-cd}
\usepackage{graphicx}
\usepackage{float}
\usepackage{wrapfig}
\usepackage[backend=biber, style=alphabetic]{biblatex}
\usepackage{import}
\usepackage{hyperref}
\usepackage{cleveref}
\usepackage[left=3cm,right=3cm,top=4.5cm,bottom=4.5cm]{geometry}

\theoremstyle{plain}
\newtheorem{thm}{Theorem}
\newtheorem{prop}[thm]{Proposition}
\newtheorem{lem}[thm]{Lemma}
\newtheorem{cor}[thm]{Corollary}
\newtheorem{exmp}[thm]{Example}
\theoremstyle{definition}
\newtheorem{defi}[thm]{Definition}
\theoremstyle{remark}
\newtheorem{rmk}[thm]{Remark}

\addbibresource{main.bib}
\newcommand{\citestacks}[1]%
{\cite[\href{https://stacks.math.columbia.edu/tag/#1}{Tag #1}]{stacks-project}}

\newcommand{\quot}[1]{“#1”}

\DeclareMathOperator{\colim}{colim}
\DeclareMathOperator{\Hom}{Hom}
\DeclareMathOperator{\rhom}{RHom}
\DeclareMathOperator{\Homs}{\underline{Hom}}
\DeclareMathOperator{\rhoms}{\underline{RHom}}
\DeclareMathOperator{\modcat}{Mod}
\newcommand{\opcat}{^{\mathrm{op}}}
\newcommand{\lprof}{\mathsf{lProFin}}
\newcommand{\huflag}{\triangleright}
\newcommand{\D}{\mathcal{D}}
\newcommand{\comp}{^{\wedge}}
\newcommand{\heart}{\heartsuit}

\newfontfamily\symbolafont{Symbola.ttf}
\newcommand{\material}{\text{\symbolafont{🜔}}}
\newcommand{\spirit}{\text{\symbolafont{☿}}}
\newcommand{\widget}{\text{\symbolafont{🍕}}}
\newcommand{\condani}{\mathsf{CondAni}}
\title{Analytic rings}
\author{}
\date{\today}

\begin{document}
\maketitle

\section*{Conventions}

Condensed always means light condensed.

I will use the \quot{animated} series of terminologies for derived objects in order not to collide with the condensed direction of the structure.
In particular,
the word \emph{discrete} is reserved for being a constant sheaf on the site of profinite sets, while
\emph{static} stands for concentrated in degree $ 0 $.
Rings are not animated unless explicitly stated to be so.

I will use the expression \emph{chain complex} to mean a differential $ \mathbb{Z} $-graded object, with
the differentials always going to the right.
I allow myself the freedom to index a chain complex with either homological degrees or cohomological degrees with the convention that
cohomological indexes are put on the top right corner while homological indexes are on the bottom right corner.
So if $ K $ is a complex, then $ K ^{i} \equiv K _{-i} $ (definitional equality).
If I ever use an index of a chain complex that is not specified to be homological or cohomological, it is a bug in the writing.

Derived ($ \infty $-)categories refer to the unbounded variant unless otherwise mentioned.
There is no gluing happening in this talk so it is safe to interprete the symbol $ \D(R) $ as either the derived $ \infty $-category
or the ordinary derived category.

\section{Definition and formal constructions}
This definition is from \cite[\href
{https://www.youtube.com/watch?v=YxSZ1mTIpaA\&list=PLx5f8IelFRgGmu6gmL-Kf\_Rl\_6Mm7juZO\&t=3962s}{Video 1, 1:06:02}]{ihesvid}.
We need the following abstract nonsense called the reflection principle:
\begin{thm}
If $ \mathcal{C} $ is a presentable $ \infty $-category and $ \mathcal{D} $ a full subcategory closed under all limits and $ \kappa $-filtered colimits
for some regular cardinal $ \kappa $ ($ \kappa = \aleph _{0} $, for example),
then $ \mathcal{D} $ is also presentable and the inclusion functor $ \mathcal{D}\to \mathcal{C} $ has a left adjoint.
\end{thm}

For proof see \cite{ragimov_infty-categorical_2022}. Also see \cite{adamek_reflections_1989} for an analogue for ordinary categories.

\begin{defi}
A \emph{pre-analytic ring} $ A $ is a pair $ (A ^{\huflag}, \D(A)) $ where $ A ^{\huflag} $ is a condensed animated ring and
$ \D (A) $ a full subcategory of $ \D (A ^{\huflag}) $ satisfying the following conditions:

\begin{enumerate}
\item The subcategory $ \D (A) $ is closed under all limits and colimits in $ \D (A ^{\huflag}) $ 
(so the inclusion functor has a left adjoint which we denote by $ - \otimes _{A ^{\huflag}} A $);
\item If $ M $ is some object of $ \D (A ^{\huflag}) $ and $ N $ is in $\D (A) $, then $ \rhoms _{R ^{\huflag}} (M, N) $ lies in $ \D (A) $;
\item The functor $ -\otimes _{A ^{\huflag}} A $, composed with the inclusion, takes $ \D ^{\leq 0}(A ^{\huflag}) $ to $ \D ^{\leq 0}(A ^{\huflag}) $;
\end{enumerate}
A pre-analytic ring $ (A ^{\huflag}, \D (A)) $ is called an \emph{analytic ring} if
the object $ A ^{\huflag} $ lies in $ \D (A) $.
A morphism of (pre-)analytic rings $ A\to B $ is a morphism $ f: A ^{\huflag}\to B ^{\huflag} $ of condensed animated rings such that when
$ M\in \D (B ^{\huflag}) $ lies in $ \D (B) $, the restriction of scalars $ M \in D (A ^{\huflag}) $ lies in $ \D (A) $.
\end{defi}

The subcategory $ \D (A) $ should be thought of as complete $ A ^{\huflag} $-modules.
If there is no risk of confusion, we would say an object $ M $ of $ \D (A ^{\huflag}) $ is \emph{complete} if it lies in $\D (A) $.
We will say that $ (A ^{\huflag}, \D (A)) $ is a structure of analytic ring on $ A ^{\huflag} $.

The following formal properties are immediate from the definition.
\begin{prop}
Let $ A = (A ^{\huflag}, \D (A)) $ be an analytic ring.
There is a unique symmetric monoidal structure on $ \D (A) $ making the functor $ -\otimes _{A ^{\huflag}} A $ symmetric monoidal.
\end{prop}

Informally the structure is defined to be $ M \otimes _{A} N := (M \otimes _{A ^{\huflag}} N)\otimes _{A ^{\huflag}} A $.

\begin{proof}

\end{proof}

\begin{prop}
Let $ A = (A ^{\huflag}, \D (A)) $ be an analytic ring.
\begin{enumerate}
\item The full subcategories $ \D ^{\leq 0}(A) := \D (A) \bigcap \D ^{\leq 0}(A ^{\huflag}) $ and $ D ^{\geq 0} := \D (A) \bigcap \D ^{\geq 0} (A ^{\huflag}) $
give a $ t $-structure of $ \D (A) $;
\item The heart $ \D (A)^{\heart} $ is a full additive subcategory of $ \modcat{\pi _{0}(A ^{\huflag})} $ closed under limits, colimits
and extensions.
\item An object $ M $ of $ \D (A ^{\huflag}) $ is complete if and only if for all $ i\in \mathbb{Z} $ the $ \pi _{0}(A ^{\huflag}) $-module
$ H ^{i}(M) $ lies in $ \D (A)^{\heart} $.
\end{enumerate}
\end{prop}

\begin{proof}
(1)
As the subcategory $ \D (A) $ is a full subcategory of $ \D (A ^{\huflag}) $ closed under all limits and colimits,
the vanishing of $ \rhoms $ from a connective object to the right shift of a coconnective object
and the closure of connective (resp. coconnective) objects under left (resp. right) shift
are inherited from the standard $ t $-structure on $ \D (A ^{\huflag}) $.

Suppose that $ M \in \D (A ^{\huflag}) $ is complete. We show first that $ \tau ^{\leq i} M $ and $ \tau ^{\geq i}M $ are both complete.
As $ \D (A) $ is closed under all limits and colimits, we only have to show that $ \tau ^{\leq 0}M $ is complete for all complete $ M $.
To avoid cubersome notations we temporarily denote the functor $ -\otimes _{A ^{\huflag}} A $ by $ L $.
Since $ M $ is complete, the natural morphism $ \tau ^{\leq 0}M\to M $ factors as $ \tau ^{\leq 0}M\to L \tau ^{\leq 0}M\to M $.
Applying $ \tau ^{\leq 0} $ to this diagram and noting that $ L $ preserves $ \D ^{\leq 0} (A ^{\huflag}) $,
we have a diagram $ \tau ^{\leq 0}M\to L \tau ^{\leq 0}\to \tau ^{\leq 0}M $ and the composition is the identity on $ \tau ^{\leq 0} M $.
This shows that $ \tau ^{\leq 0}M $ is a retract of $ L \tau ^{\leq 0}M $ which is complete.
As $ \D (A) $ is closed under colimits, $ \tau ^{\leq 0}M $ is also complete.

For every complete $ M\in \D (A ^{\huflag}) $ there is a fiber sequence
$ \tau ^{\leq 0}M\to M\to \tau ^{\geq 1}M $ in $ \D (A ^{\huflag}) $.
We have shown that $ \tau ^{\leq 0}M $ and $ \tau ^{\geq 1}M $ both belong to $ \D (A) $.
Applying $ L $ which preserves colimits, we have a fiber sequence
$ \tau ^{\leq 0}M\to M\to \tau ^{\geq 1}M $ in $ \D (A) $
with $ \tau ^{\leq 0}M $ living in $ \D ^{\leq 0}(A) $ and $ \tau ^{\geq 1}M $ in $ \D ^{\geq 1}(A) $.
So the pair $ (\D ^{\leq 0}(A), \D ^{\geq 1}(A)) $ does give a $ t $-structrure on $ \D (A) $.

(2)
By definition the heart $ \D (A)^{\heart} $ is a full subcategory of $ \D (A ^{\huflag})^{\heart} $, which is canonically equivalent to 
$ \D (\pi _{0}(A ^{\huflag})) $.

\end{proof}

When $ A $ is an analytic ring with $ A ^{\huflag} $ being static,
we call $ \D(A)^{\heart} $ the category of $ A $-modules.
Note that $ \D (A) $ is not neccessarily the derived $ \infty $-category of $ \D (A)^{\heart} $.
(even if $ A $ is static?)

\begin{prop}
Let $ A = (A ^{\huflag}, \D (A)) $ be an analytic ring.
If $ M $ is an object in $ \D (A ^{\huflag}) $, then $ M $ lies in $ \D (A) $ if and only if all $ H ^{i}(M) $ lies in $ \D (A)^{\heart} $.
\end{prop}

\begin{proof}

\end{proof}

\begin{exmp}
For any condensed ring $ R $, the pair $ (R, \D (R)) $ is an analytic ring. We call this the trivial analytic ring structure on $ R $.
\end{exmp}

\begin{exmp}
Let $ (A ^{\huflag}, \D (A)) $ be a pair satisfying all but condition (4) in the definition of an analytic ring.
We call such a pair a \emph{pre-(analytic ring)}.
There is a recipe for completing a pre-(analytic ring) into an analytic ring.
Let $ A ^{\trianglerighteq} $ be the $ A ^{\huflag} $-module $ A ^{\huflag} \otimes _{A ^{\huflag}} A $.
\end{exmp}

\begin{exmp}
Let $ A ^{\huflag}\to B ^{\huflag} $ be a morphism of condensed rings and $ (A ^{\huflag}, \D (A)) $ a structure of analytic ring on $ A ^{\huflag} $.
Define $ \D (B) $ to be the full subcategory of $ \D (B) $ spanned by objects $ M $ such that the restriction of scalar of $ M $ to $ \D (A ^{\huflag}) $
lies in $ \D (A) $. We claim that $ (B ^{\huflag}, \D(B)) $ is a structure of analytic ring on $ B ^{\huflag} $.

Indeed, the restriction of scalar functor $ \D (B ^{\huflag})\to \D (A ^{\huflag}) $ has both left and right adjoints, so it commutes with all limits and colimits.
This implies that $ \D (B) $ is closed under all limits and colimits.
If $ M $ lies in $ \D (B) $ and $ N $ is some object of $ \D (B ^{\huflag}) $,
then the restriction of scalar of $ \rhoms _{B ^{\huflag}}(N, M) $ to $ \D (A ^{\huflag}) $
\end{exmp}
\appendix
\section{Condensed Anima}

The notion of condensed animated sets/abelian groups/rings behave very well in the old condensed framework of \cite{condensed},
as the old category of condensed sets/abelian groups/rings are generated by compact projective objects of the form $ \mathbb{Z}[S] $
where $ S $ is extremally disconnected.
However, we lose most of these projective objects in the light setting.
Thus we need to put more care into defining condensed derived objects.

We start with the $ \infty $-topos where all condensed derived objects live.
\begin{defi}
A \emph{condensed anima} is a hypersheaf of anima on the site $ \mathrm{Pro}_{\mathbb{N}}(\mathsf{FinSet}) $.
The $ \infty $-category of condensed anima is denoted by $\condani$.
\end{defi}

A sequential limit of coverings in the site $ \mathrm{Pro}_{\mathbb{N}}(\mathsf{FinSet}) $ is also a covering.
Thus the topos of condensed sets is replete.
A result of Mondal and Reinecke in \cite{mondal_postnikov_2024} shows that $ \condani $ is Postnikov complete.

\begin{defi}
If \widget{} is an algebraic structure, then
a \emph{condensed animated \widget{}} is a \widget -object in $ \condani $.
The $ \infty $-category of condensed animated \widget{} is denoted by $ \condani\widget $.
\end{defi}

If \widget{} is abelian groups, there is another reasonable way to define condensed animated abelian groups:
the category $ \mathsf{CondAb} $ is a Grothendieck abelian category and we can construct its (unbounded) derived $ \infty $-category
using the injective model structure on the category of chain complexes.
Fortunately the two approaches agree.
\begin{prop}
There is an equivalence $ \D (\mathsf{CondAb})\to \mathrm{HypSh}(\mathrm{Pro}_{\mathbb{N}}(\mathsf{FinSet}), \D (\mathbb{Z})) $.
\end{prop}

\begin{proof}
There is a $ t $-structure on the right hand side
where an object $ X $ is connective if and only if all sections $ X (U) $ on light profinite sets are connective.
The heart of this $ t $-structure is $ \mathsf{CondAb} $.
Then we can
apply \cite{sag}, Theorem 2.1.2.2 to the $ \infty $-topos $ \condani $ with $ \mathcal{O} = \mathbb{Z} $.
\end{proof}

Note that the proof relies on the hypercomplete-ness of $ \mathsf{CondAni} $.



\printbibliography
\end{document}

