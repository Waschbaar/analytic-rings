\documentclass{article}

\usepackage{fontspec}
\usepackage{amsmath}
\usepackage{amsthm}
\usepackage{amssymb}
\usepackage{amsfonts}
\usepackage{mathrsfs}
\usepackage{tikz}
\usepackage{tikz-cd}
\usepackage{graphicx}
\usepackage{float}
\usepackage{wrapfig}
\usepackage[backend=biber, style=alphabetic]{biblatex}
\usepackage{import}
\usepackage{hyperref}
\usepackage{cleveref}
\usepackage{marginnote}
\usepackage[left=3cm,right=3cm,top=4.5cm,bottom=4.5cm]{geometry}

\theoremstyle{plain}
\newtheorem{thm}{Theorem}
\newtheorem{prop}[thm]{Proposition}
\newtheorem{lem}[thm]{Lemma}
\newtheorem{cor}[thm]{Corollary}
\newtheorem{exmp}[thm]{Example}
\theoremstyle{definition}
\newtheorem{defi}[thm]{Definition}
\theoremstyle{remark}
\newtheorem{rmk}[thm]{Remark}

\addbibresource{main.bib}
\newcommand{\citestacks}[1]%
{\cite[\href{https://stacks.math.columbia.edu/tag/#1}{Tag #1}]{stacks-project}}

\newcommand{\quot}[1]{“#1”}

\DeclareMathOperator{\colim}{colim}
\DeclareMathOperator{\Hom}{Hom}
\DeclareMathOperator{\rhom}{RHom}
\DeclareMathOperator{\Homs}{\underline{Hom}}
\DeclareMathOperator{\rhoms}{\underline{RHom}}
\DeclareMathOperator{\modcat}{Mod}
\DeclareMathOperator{\ext}{Ext}
\DeclareMathOperator{\exts}{\underline{Ext}}
\newcommand{\dten}{\otimes ^{L}}
\newcommand{\opcat}{^{\mathrm{op}}}
\newcommand{\lprof}{\mathsf{lProFin}}
\newcommand{\huflag}{\triangleright}
\newcommand{\D}{\mathcal{D}}
\newcommand{\comp}{^{\wedge}}
\newcommand{\heart}{\heartsuit}
\newcommand{\solid}{\square}

\newcommand{\footwrite}[2]{%
\footnote{%
{\color{#1} #2}%
}}

\newcommand{\bullshit}[1]{\footnote{#1}}

\definecolor{respcolor}{RGB}{99,140,109}
\newcommand{\resp}[1]{{\color{respcolor}(resp. #1)}}

\newfontfamily\symbolafont{Symbola.ttf}
\newcommand{\material}{\text{\symbolafont{🜔}}}
\newcommand{\spirit}{\text{\symbolafont{☿}}}
\newcommand{\widget}{\text{\symbolafont{🍕}}}
\newcommand{\condani}{\mathsf{CondAni}}
\title{Analytic rings}
\author{}
\date{\today}

\begin{document}
\maketitle
\tableofcontents
\pagebreak

\section*{Conventions}

Set theoretic difficulties are completely ignored.
I promise that I did not invoke Russell's paradox.
\bullshit{ZFC's consistency is (provably) only a belief.}

Profinite always means light profinite. Condensed always means light condensed.

I will use the \emph{animated} series of terminologies for derived objects in order not to collide with the condensed direction of the structure.
In particular,
the word \emph{discrete} is reserved for being a constant sheaf on the site of profinite sets, while
\emph{static} stands for concentrated in degree $ 0 $.
Rings are not animated unless explicitly stated to be so.

I will use the expression \emph{chain complex} to mean a differential $ \mathbb{Z} $-graded object, with
the differentials pointing to the right.
I allow myself the freedom to index a chain complex with either homological degrees or cohomological degrees with the convention that
cohomological indexes are put on the top right corner while homological indexes are on the bottom right corner.
So if $ K $ is a complex, then $ K ^{i} \equiv K _{-i} $ (definitional equality).
Similar convention goes for the (co)homology so $ \mathrm{Tor}^{-1}\equiv \mathrm{Tor}_{1} $.
If I ever use an index of a chain complex that is not specified to be homological or cohomological, it is a bug in the writing.

Derived ($ \infty $-)categories refer to the unbounded variant unless otherwise mentioned.
There is no gluing happening in this talk so it is safe to interprete the symbol $ \D(R) $ as either the derived $ \infty $-category
or the ordinary derived category.
The term \emph{isomorphism} used in derived categories means a quasi-isomorphism.

\pagebreak

\section{Definition}

The following definition is given in
\cite[\href{https://www.youtube.com/watch?v=dIwBTJNN7a0\&list=PLx5f8IelFRgGmu6gmL-Kf\_Rl\_6Mm7juZO\&index=8\&t=1633s}{Video 8, 27:13}]{ihesvid}.
\begin{defi}
A \emph{pre-analytic ring} is a pair $ (A ^{\huflag}, \modcat (A)) $ where $ A $ is a condensed ring and
$ \modcat (A) $ a full additive subcategory of $ \modcat (A ^{\huflag}) $ satisfying the following conditions:

\begin{enumerate}
\item The inclusion functor $ \modcat (A)\to \modcat (A ^{\huflag}) $ has a left adjoint
, which we denote by $ -\otimes _{A ^{\huflag}} A$, and a right adjoint, which we denote by $ \Homs _{A ^{\huflag}}(A, -) $;
\item The subcategory $ \modcat (A) $ is closed under extensions;
\item If $ M,N $ are $ A ^{\huflag} $-modules and $ N $ lies in $ \modcat (A) $,
then $ \exts ^{i} _{A ^{\huflag}} (M, N) $ lies in $ \modcat (A) $ for all $ i\geq 0 $.
\end{enumerate}
We denote
A pre-analytic ring is called an \emph{analytic ring} if the object $ A ^{\huflag} $ lies in $ \modcat (A) $.

A morphism of (pre-)analytic rings $ A\to B $ is a morphism $ f: A ^{\huflag}\to B ^{\huflag} $ such that
when $ M $ lies in $ \modcat (B) $, the restriction of scalars of $ M $ lies in $ \modcat (A) $.
\end{defi}

There is also a derived version of the definition.
This definition is a modified but equivalent version of \cite[\href
{https://www.youtube.com/watch?v=YxSZ1mTIpaA\&list=PLx5f8IelFRgGmu6gmL-Kf\_Rl\_6Mm7juZO\&t=3962s}{Video 1, 1:06:02}]{ihesvid}.

\begin{defi}
A \emph{pre-analytic animated ring} $ A $ is a pair $ (A ^{\huflag}, \D(A)) $ where $ A ^{\huflag} $ is a condensed animated ring and
$ \D (A) $ a full subcategory of $ \D (A ^{\huflag}) $ satisfying the following conditions:

\begin{enumerate}
\item The inclusion functor $ \D (A)\to \D (A ^{\huflag}) $ has a left adjoint, which we denote by $ -\dten _{A ^{\huflag}} A $,
and a right adjoint, which we denote by $ \rhoms _{A ^{\huflag}}(A, -) $;
\item If $ M,N $ are some objects of $ \D (A ^{\huflag}) $ such that $ N $ is in $\D (A) $, then $ \rhoms _{A ^{\huflag}} (M, N) $ lies in $ \D (A) $;
\item The functor $ -\dten _{A ^{\huflag}} A $, composed with the inclusion, takes $ \D ^{\leq 0}(A ^{\huflag}) $ to $ \D ^{\leq 0}(A ^{\huflag}) $;
\end{enumerate}
A pre-analytic animated ring $ (A ^{\huflag}, \D (A)) $ is called an \emph{analytic animated ring} if
the object $ A ^{\huflag} $ lies in $ \D (A) $.

A morphism of (pre-)analytic animated rings $ A\to B $ is a morphism $ f: A ^{\huflag}\to B ^{\huflag} $ of condensed animated rings such that when
$ M\in \D (B ^{\huflag}) $ lies in $ \D (B) $, the restriction of scalars $ M \in D (A ^{\huflag}) $ lies in $ \D (A) $.
\end{defi}

The subcategory $ \modcat (A) $ (or $ \D (A) $) should be thought of as complete $ A ^{\huflag} $-modules.
If there is no risk of confusion, we would say an object $ M $ of $ \modcat (A) $ \resp{$ \D (A ^{\huflag}) $} is
\emph{complete} if it lies in $ \modcat (A) $ \resp{$\D (A) $}.
We will say that $ (A ^{\huflag}, \modcat (A)) $ \resp{$ (A ^{\huflag}, \D (A)) $} is a structure of a (pre-)analytic \resp{animated} ring on $ A ^{\huflag} $.

\begin{rmk}
\label{definition::nonsense}
Instead of requiring the inclusion functor to have both left and right adjoints,
Clausen and Scholze require it to commute with all limits and colimits.
The two are equivalent in view of the reflection principle,
which is proved in \cite{adamek_reflections_1989} for ordinary categories and in \cite{ragimov_infty-categorical_2022} for $ \infty $-categories.

Indeed, when the subcategory $ \D (A) $ (resp. $ \modcat (A) $) is closed under arbitrary limits and colimits,
the reflection principle implies that the inclusion functor $ \D (A)\to \D (A ^{\huflag}) $ \resp{$ \modcat (A)\to \modcat (A ^{\huflag}) $}
has a left adjoint and that the subcategory
$ \D (A) $ \resp{$ \modcat (A) $} is presentable.
Then the adjoint functor theorem between presentable ($ \infty $-)categories implies the existence of the right adjoint.
Conversely, if the inclusion functor has both left and right adjoints, then the subcategory is presentable as
a localization of a presentable ($ \infty $-)category. Then it has all limits and colimits,
which are taken to limits and colimits in $ \D (A) $ (resp. $ \modcat (A) $) since the inclusion functor
is both a left and a right adjoint.

The definition of Clausen and Scholze is certainly more natural.
However I decide to bake the adjoints into the definition for the peace of mind
\bullshit{I don't care about set theory but I did promise to not invoke Russell's paradox.},
even though I failed to avoid invoking the reflection principle (and the adjoint functor theorem) in the sequel.
\end{rmk}

\subsection{Comparison derived vs. non-derived}
The comparison works in much more generality than is proved here, see Remark \ref{definition::animated}.

\begin{lem}
Let $ (A ^{\huflag}, \D (A)) $ be a pre-analytic animated ring.
The subcategory $ \D (A)\subset \D (A ^{\huflag}) $ satisfies 2-out-of-3 in distinguished triangles.
\end{lem}

\begin{proof}
The subcategory $ \D (A) $ is closed under limits and colimits, so in particular fibers and cofibers.

Suppose that $ M'\to M\to M'' $ is a distinguished triangle where $ M',M'' $ lies in $ \D (A) $.
Then $ M\to M''\to M'[1] $ is also a fiber sequence, and $ M' [1] $ lies in $ \D (A) $ since left shifting is a colimit.
This expresses $ M $ as a fiber of a map between objects in $ \D (A) $, so $ M $ lies in $ \D (A) $.
\end{proof}

\begin{prop}
Let $ A = (A ^{\huflag}, \D (A)) $ be a pre-analytic animated ring where $ A ^{\huflag} $ is static.
\begin{enumerate}
\item The subcategory $ \D (A) $ is closed under truncations in $ \D (A ^{\huflag}) $;
\item The full subcategories $ \D ^{\leq 0}(A) := \D (A) \bigcap \D ^{\leq 0}(A ^{\huflag}) $ and $ D ^{\geq 0} := \D (A) \bigcap \D ^{\geq 0} (A ^{\huflag}) $
give a $ t $-structure of $ \D (A) $;
\item The functor $ \rhoms _{A ^{\huflag}}(A, -) $ preserves $ \D ^{\geq 0}(A ^{\huflag}) $;
\item The heart $ \D (A)^{\heart} $ is a full additive subcategory of $ \modcat (A ^{\huflag}) $ closed under extensions,
and the inclusion functor $ \D (A)^{\heart}\to \modcat (A ^{\huflag}) $ has both left and right adjoints;
\item An object $ M $ of $ \D (A ^{\huflag}) $ is complete if and only if for all $ i\in \mathbb{Z} $ the $ A ^{\huflag} $-module
$ H ^{i}(M) $ lies in $ \D (A)^{\heart} $.
\end{enumerate}
\label{definition::tstr}
\end{prop}

\begin{proof}
(1)
Suppose that $ M \in \D (A ^{\huflag}) $ is complete. We need to show that $ \tau ^{\leq i} M $ and $ \tau ^{\geq i}M $ are both complete.
As $ \D (A) $ is closed under all limits and colimits, we only have to show that $ \tau ^{\leq 0}M $ is complete for all complete $ M $.
To avoid cumbersome notations we temporarily denote the functor $ -\dten _{A ^{\huflag}} A $ by $ L $.
Since $ M $ is complete, the natural morphism $ \tau ^{\leq 0}M\to M $ factors as $ \tau ^{\leq 0}M\to L \tau ^{\leq 0}M\to M $.
Applying $ \tau ^{\leq 0} $ to this diagram and noting that $ L $ preserves $ \D ^{\leq 0} (A ^{\huflag}) $,
we have a diagram $ \tau ^{\leq 0}M\to L \tau ^{\leq 0}\to \tau ^{\leq 0}M $ and the composition is the identity on $ \tau ^{\leq 0} M $.
This shows that $ \tau ^{\leq 0}M $ is a retract of $ L \tau ^{\leq 0}M $ which is complete.
As $ \D (A) $ is closed under colimits, $ \tau ^{\leq 0}M $ is also complete.

(2)
As the subcategory $ \D (A) $ is a full subcategory of $ \D (A ^{\huflag}) $ closed under all limits and colimits,
the vanishing of $ \rhoms $ from a connective object to the right shift of a coconnective object
and the closure of connective \resp{coconnective} objects under left \resp{right} shift
are inherited from the standard $ t $-structure on $ \D (A ^{\huflag}) $.

For every complete $ M\in \D (A ^{\huflag}) $ there is a fiber sequence
$ \tau ^{\leq 0}M\to M\to \tau ^{\geq 1}M $ in $ \D (A ^{\huflag}) $.
We have shown that $ \tau ^{\leq 0}M $ and $ \tau ^{\geq 1}M $ both belong to $ \D (A) $.
Applying $ -\dten _{A ^{\huflag}} A $ which preserves colimits, we have a fiber sequence
$ \tau ^{\leq 0}M\to M\to \tau ^{\geq 1}M $ in $ \D (A) $
with $ \tau ^{\leq 0}M $ living in $ \D ^{\leq 0}(A) $ and $ \tau ^{\geq 1}M $ in $ \D ^{\geq 1}(A) $.
So the pair $ (\D ^{\leq 0}(A), \D ^{\geq 1}(A)) $ does give a $ t $-structrure on $ \D (A) $.

(3)
We showed in (2) that $ (\D ^{\leq 0}(A), \D ^{\geq 1}(A)) $ is a $ t $-structure.
So it suffices to check that $$ \Hom _{A ^{\huflag}}(M, \Homs _{A ^{\huflag}}(A, N)) = 0 $$
for all $ N\in \D ^{\geq 0}(A ^{\huflag}) $ and $ M\in \D ^{\leq -1}(A) $.
But this can be calculated by adjunction.

(4)
By definition the heart $ \D (A)^{\heart} $ is a full subcategory of $ \D (A ^{\huflag})^{\heart} $, which is canonically equivalent to 
$ \modcat (A ^{\huflag}) $.
The left and right adjoints of the inclusion $ \D (A)\to \D (A ^{\huflag}) $
preserve $ \D ^{\leq 0}(A) $ and $ \D ^{\geq 0}(A) $ respectively.
Thus their composition with $ H ^{0} $ give left and right adjoints of $ \D (A)^{\heart}\to \modcat (A ^{\huflag}) $.
Exact sequences in $ \modcat (A ^{\huflag}) $ give rise to distinguished triangles in $ \D (A ^{\huflag}) $ and
the subcategory $ \D (A) $ has 2-out-of-3 property in fiber sequences. Thus the extension of two objects in $ \D (A) ^{\heart} $
still lies in $ \D (A) ^{\heart} $.

(5)
If $ M $ is complete then so is all $ H ^{i}(M) $ thanks to (1).
Conversely, suppose that $ M $ is a object of $ \D (A ^{\huflag}) $ such that all $ H ^{i}(M) $ lies in $ \D (A)^{\heart} $.
We first show by induction that
if such an $ M $ is bounded ,i.e. $ M $ lies in some $ \D ^{\leq m}(A ^{\huflag}) \bigcap \D ^{\geq n}(A ^{\huflag}) $,
then $ M $ is complete.
There is nothing to show if $ m-n=0 $. Suppose that the statement is true for all $ m,n $ such that $ m-n\leq k $,
and let $ M $ be an object of $ \D ^{\leq m}(A ^{\huflag}) \bigcap \D ^{\geq n}(A ^{\huflag}) $ such that $ m-n=k+1 $.
There is a distinguished triangle $ \tau ^{\leq m-1} M \to M \to \tau ^{\geq m} M $.
The object $ \tau ^{\geq m}M $ lies in $ \D (A ^{\huflag})^{\heart}[-m] $, so it is complete.
The object $ \tau ^{\leq m-1} M $ is complete by induction hypothesis.
So $ M $ is complete.
If $ M $ is bounded on the left, then we may write $ M $ as a colimit of its truncations to cohomological degrees $ \leq n $.
Each of the truncations is complete, and $ \D (A) $ is closed under colimits, so such an $ M $ is complete.
In general, $ M $ is a limit of its truncations to cohomological degrees $ \geq n $ since $ \D (A ^{\huflag}) $
is Postnikov complete.
Each of the truncations is complete by the arguments above, and thus $ M $ is complete since $ \D (A) $ is closed under limits.
\end{proof}

\begin{prop}
Let $ A ^{\huflag} $ be a condensed ring, which is also regarded as a static condensed animated ring.
There is a bijective correspondence
$$ \{\text{pre-analytic animated ring structures on } A ^{\huflag}\} \to \{\text{pre-analytic ring strcutures on } A ^{\huflag}\} $$
taking a full subcategory $ \D (A)\subset \D (A ^{\huflag}) $ to the heart of the $ t $-structure in Proposition \ref{definition::tstr}.
The inverse sends a subcategory $ \modcat (A)\subset \modcat (A ^{\huflag}) $ to the full subcategory of $ \D (A ^{\huflag}) $ spanned by
objects $ M $ such that $ H ^{i}(M) $ lies in $ \modcat (A) $ for every $ i\in \mathbb{Z} $.

Analytic animated ring structures correspond to analytic ring structures under the bijection.
\label{definition::comparison}
\end{prop}

\begin{proof}
If $ \D (A) $ defines a pre-analytic animated ring structure on $ A ^{\huflag} $, then Proposition \ref{definition::tstr}
shows that $ \D (A)^{\heart} $ defined a pre-analytic ring structure on $ A $,
and that $ \D (A) $ is recovered from $ \D (A)^{\heart} $ by the inverse map.

Conversely, suppose that $ \modcat (A) $ defines a pre-analytic ring strcuture on $ A ^{\huflag} $.
Let $ \D (A) $ be the subcategory of $ \D (A ^{\huflag}) $ determined by the inverse.
By definition the abelian category $ \D (A) \bigcap \D (A ^{\huflag})^{\heart} $ is $ \modcat (A) $.
It remains to check that $ \D (A) $ defines a pre-analytic animated ring structure on $ A ^{\huflag} $.

Using the cohomological long exact sequence we see that $ \D (A) $ satisfies 2-out-of-3 in distinguished triangles.
Arbitrary direct sums are exact in $ \modcat (A ^{\huflag}) $, and thus commuting with $ H ^{i} $ on $ \D (A ^{\huflag}) $.
So $ \D (A) $ is closed under arbitrary direct sums.
Now let $ \{M _{i}\}_{i\in \mathcal{I}} $ be a collection of objects in $ \D (A) $.
Denote by $ M $ the direct sum of all $ M _{i} $.
Then each $ M _{i} $ is a retract of $ M $, so $ H ^{i}(\prod _{i\in \mathcal{I}} M _{i}) $
is a retract of $ H ^{i}(\prod _{i\in \mathcal{I}} M) $.
But $ H ^{i}(\prod _{i\in \mathcal{I}} M) $ can also be calculated as $ \exts ^{i}(\bigoplus _{i\in \mathcal{I}} A ^{\huflag}, M) $,
which lies in $ \modcat (A) $. So $ \prod _{i\in \mathcal{I}} M _{i} $ lies in $ \D (A) $.
As homotopy limits and colimits can be calculated using direct sums, direct products and distinguished triangles,
we conclude that $ \D (A) $ is closed under limits and colimits.
Then we use the reflection principle in Remark \ref{definition::nonsense} to find the left and right adjoints of $ \D (A)\to \D (A ^{\huflag}) $.

Now suppose that $ M,N $ are objects in $ \D (A ^{\huflag}) $ and that $ N $ lies in $ \D (A) $.
We need to show that $ \rhoms _{A ^{\huflag}}(M, N) $ lies in $ \D (A) $.
If $ M $ is bounded to the right and $ N $ is bounded to the left, then we may use a spectral sequence to calculate the cohomology of
$ \rhoms _{A ^{\huflag}}(M, N) $ in terms of kenerls, cokernels and extensions of various $ \exts ^{i}(H ^{p}(M), H ^{q}(N)) $.
Then $ \rhoms _{A ^{\huflag}(M, N)} $ lies in $ \D (A) $.
The general case follows by writing $ M $ as a colimit of its right bounded truncations and $ N $ as a limit of its left bounded truncations.

Now we need to check that if $ M\in \D (A ^{\huflag}) $ is connective, then $ LM $ is also connective, where $ L $ is the left adjoint
of $ \D (A)\to \D (A ^{\huflag}) $.
As $ \D (A) $ is clearly closed under truncations, $ \tau ^{\leq 0} L M $ lies in $ \D (A) $.
There is a map $ M\to \tau ^{\leq 0}LM $ since $ M $ is connective.
But then $ \tau ^{\leq 0}LM $ satisfies the unversal property of $ LM $,
so $ \tau ^{\leq 0}LM\simeq LM $.
This finishes the proof that $ \D (A) $ defines a pre-analytic ring structure.

When a structure of a pre-analytic animated ring \resp{pre-analytic ring} on $ A ^{\huflag} $ is given,
the remaining requirement for it to be an analytic animated ring \resp{analytic ring} is that $ A ^{\huflag} $ lies in
$ \D (A) $ \resp{$ \modcat (A) $}. The conditions in derived and non-derived situations coincide since $ A ^{\huflag} $ is static.
\end{proof}

\begin{rmk}
The bijection is functorial with respect to the morphisms of pre-analytic (animated) rings.

Indeed, if $ (A ^{\huflag}, \D (A))\to (B ^{\huflag}, \D (B)) $ is a morphism of analytic animated rings
where $ A ^{\huflag}, B ^{\huflag} $ are static,
then the restriction of scalar of objects in $ \D (B) ^{\heart} $ lies in $ \D (A) $,
and so in $ \D (A)^{\heart} $ since restriction of scalar commutes with truncations.
Then $ (A ^{\huflag}, \D (A)^{\heart})\to (B ^{\huflag}, \D (B)^{\heart}) $ is a morphism of pre-analytic rings.
The reverse direction is similar.
\end{rmk}

\begin{rmk}
If $ A ^{\huflag} $ is a condensed animated ring, then the same maps define a bijection between pre-analytic animated ring structures on
$ A ^{\huflag} $ and pre-analytic ring structures on $ \pi _{0}(A ^{\huflag}) $.
(In this case an analytic animated ring is mapped to an analytic ring, but not conversely since completeness needs to be tested on all $ \pi _{i} $.)
This is stated by Clausen in
\cite[\href{https://www.youtube.com/watch?v=38PzTzCiMow\&list=PLx5f8IelFRgGmu6gmL-Kf\_Rl\_6Mm7juZO\&index=13\&t=2099s}{Video 13, 34:59}]{ihesvid}.
Clausen pointed to \cite{analytic} for a proof, which I do not understand.
\label{definition::animated}
\end{rmk}

\section{Formal constructions}

\subsection{Symmetric monoidal structure}
\begin{prop}
Let $ A = (A ^{\huflag}, \D (A)) $ \resp{$ (A ^{\huflag}, \modcat (A)) $} be a pre-analytic animated ring \resp{a pre-analytic ring}.
There is a unique symmetric monoidal structure on $ \D (A) $ \resp{$ \modcat (A) $} making the functor
$ -\dten _{A ^{\huflag}} A $ \resp{$ -\otimes _{A ^{\huflag}} A $} symmetric monoidal.
\end{prop}

We write the proof only in the derived version to avoid using too many \resp{}.
The proof can be adapted to the non-derived situation with minimal modification.

Denote the symmetric monoidal structure on $ \D (A) $ by $ -\dten _{A} - $.
The requirement that $ -\dten _{A ^{\huflag}} A $ is symmetric monoidal forces us to define
$ M \dten _{A} N := (M \dten _{A ^{\huflag}} N)\dten _{A ^{\huflag}} A $.

\begin{proof}
We denote the functor $ -\dten _{A ^{\huflag}} A $ by $ L $ in this proof.

Suppose that $ f: M\to M' $ is a map in $ \D (A ^{\huflag}) $
such that $ Lf $ is an isomorphism, and that $ N $ is some object of $ \D (A ^{\huflag}) $.
For every object $ T $ of $ \D (A) $, we have that
\begin{equation*}
\rhom _{A ^{\huflag}}(M \dten _{A ^{\huflag}} N, T) \simeq \rhom _{A ^{\huflag}}(M, \rhoms _{A ^{\huflag}}(N, T))
\end{equation*}
and similarly for $ M' $.
By assumption the object $ \rhoms _{A ^{\huflag}}(N, T) $ lies in $ \D (A) $.
Thus the map
\begin{equation*}
\rhom _{A ^{\huflag}}(M' \dten _{A ^{\huflag}} N, T) \to \rhom _{A ^{\huflag}}(M \dten _{A ^{\huflag}} N, T)
\end{equation*}
induced by $ f $ is an isomorphism.
Since this is true for all $ T\in \D (A) $, we conclude that $ f $ induces an isomorphism
$ L(M\dten _{A ^{\huflag}} N)\to L(M' \dten _{A ^{\huflag}} N) $.
Then \cite{ha}, Proposition 2.2.1.9 implies that $ \D (A) $ inherits a symmetric monoidal structure making functor $ L $ symmetric monoidal.
\end{proof}

\begin{prop}
Let $ A\to B $ be a morphism of pre-analytic animated rings. Then the composition
\begin{equation*}
\D (A ^{\huflag}) \xrightarrow{- \dten _{A ^{\huflag}} B ^{\huflag}} \D (B ^{\huflag}) \xrightarrow{-\dten _{B ^{\huflag}} B} \D (B)
\end{equation*}
factors through $ \D (A ^{\huflag}) \xrightarrow{-\dten _{A ^{\huflag}} A}\D (A) $.
The factorization gives a symmetric monoidal functor $ -\dten _{A} B: \D (A)\to \D (B) $
which is the left adjoint of the restriction of scalar functor $ \D (B)\to \D (A) $.
\end{prop}

\begin{proof}
Denote the functor $ -\dten _{A ^{\huflag}} A $ by $ L _{A} $ and $ -\dten _{B ^{\huflag}} B $ by $ L _{B} $ in this proof.

Suppose that $ f: M\to N $ is a map in $ \D (A ^{\huflag}) $ such that $ L _{A} f $ is an isomorphism.
If $ T\in \D (B) $, then
\begin{equation*}
\rhom _{B ^{\huflag}}(L _{B}(M\dten _{A ^{\huflag}} B ^{\huflag}), T)\simeq
\rhom _{B ^{\huflag}}(M\dten _{A ^{\huflag}} B ^{\huflag}, T)\simeq 
\rhom _{A ^{\huflag}}(M, T)
\end{equation*}
and similarly for $ N $. By assumption (the restriction of scalars from $ B ^{\huflag} $ to $ A ^{\huflag} $ of) $ T $ lies in $ \D (A) $.
Thus the map
\begin{equation*}
\rhom _{B ^{\huflag}}(L _{B}(N\dten _{A ^{\huflag}} B ^{\huflag}), T)\to \rhom _{B ^{\huflag}}(L _{B}(M\dten _{A ^{\huflag}} B ^{\huflag}), T)
\end{equation*}
induced by $ f $ is an isomorphism.
Then the composition $ L _{B} (-\dten _{A ^{\huflag}} B ^{\huflag}) $ factors through the localization functor $ L _{A}: \D (A ^{\huflag})\to \D (A) $.

If $ M\in \D (A) $, then $ M\simeq L _{A} M $,
so we may calculate $ M \dten _{A} B $ as $ L _{B}(M \dten _{A ^{\huflag}} B ^{\huflag}) $.
Thus, if $ M,N\in \D (A) $, we may calculate that
\begin{align*}
(M\dten _{A} B)\dten _{B} (N\dten _{A} B)
&\simeq L _{B} ( (M \dten _{A ^{\huflag}} B ^{\huflag}) \dten _{B ^{\huflag}} (N \dten _{A ^{\huflag}} B ^{\huflag}))\\
&\simeq L _{B} ( (M \dten _{A ^{\huflag}} N) \dten _{A ^{\huflag}} B ^{\huflag} )\\
&\simeq L _{B} ( L _{A}(M \dten _{A ^{\huflag} N}) \dten _{A ^{\huflag}} B ^{\huflag})\\
&\simeq (M\dten _{A} N) \dten _{A} B.
\end{align*}
The isomorphisms are natural, so $ -\dten _{A} B $ is symmetric monoidal.

Finally suppose that $ M\in \D (A) $ and $ N \in \D (B) $. We calculate that
\begin{align*}
\rhom _{B ^{\huflag}}(M \dten _{A} B, N)
&\simeq \rhom _{B ^{\huflag}}(M \dten _{A ^{\huflag}} B ^{\huflag}, N)\\
&\simeq \rhom _{A ^{\huflag}}(M, N).
\end{align*}
The isomorphisms are natural so $ -\dten _{A} B $ is the left adjoint of the restriction of scalars $ \D (B)\to \D (A) $.
\end{proof}

Beware that $ -\dten _{A} B $ does not neccessarily preserve the heart even if $ A,B $ are static.

\subsection{Induced pre-analytic structure}

A lot of examples of (pre-)analytic rings arises from a certain base ring through the following construction.
Note that it is easier to do the construction on the derived level.
\begin{prop}
Let $ (A ^{\huflag}, \D (A)) $ be a pre-analytic animated ring and $ A ^{\huflag}\to B ^{\huflag} $ be a map of condensed animated rings.
Let $ \D (B) $ be the full subcategory of $ \D (B ^{\huflag}) $ spanned by the objects whose restriction of scalars to $ \D (A ^{\huflag}) $
lies in $ \D (A) $. Then the pair $ (B ^{\huflag}, \D (B)) $ is a pre-analytic animated ring.
\end{prop}

\begin{proof}
Since the restriction of scalar functor commutes with limits and colimits, the full subcategory
$ \D (B) $ is closed under limits and colimits.
Then Remark \ref{definition::nonsense} gives the required left and right adjoints.

Now suppose that $ M, N\in \D (B ^{\huflag}) $ and $ N $ lies in $ \D (B) $.
We would like to show that $ \rhoms _{B ^{\huflag}}(M, N) $ lies in $ \D (B) $.
If $ M \simeq M_0 \dten _{A ^{\huflag}} B ^{\huflag} $ for some $ M _{0}\in \D (A ^{\huflag}) $,
then
\begin{equation*}
\rhoms _{B ^{\huflag}}(M, N)\simeq \rhoms _{A ^{\huflag}}(M_0, N)
\end{equation*}
as $ A ^{\huflag} $-modules.
In this case the object $ \rhoms _{B ^{\huflag}}(M, N) $ lies in $ \D (B) $.
For general $ M $, let $ M ^{\bullet}\to M $ be the Bar resolution. Each $ M ^{i} $ is of the form $ M ^{i}_{0}\dten _{A ^{\huflag}}B ^{\huflag} $
so $ \rhoms _{B ^{\huflag}}(M ^{i}, N) $ lies in $ \D (B) $.
The object $ \rhoms _{B ^{\huflag}}(M, N) $ is then the limit of $ \rhoms _{B ^{\huflag}}(M ^{\bullet}, N) $,
which lies in $ \D (B) $ since the latter is closed under limits.

Finally we need to check that the composition of the completion and the inclusion
$ \D (B ^{\huflag})\to \D (B)\to \D (B ^{\huflag}) $ preserves the connective part.
We only need to show that $ \tau ^{\leq 0} $ preserves $ \D (B) $, by the same argument as in the proof of Proposition
\ref{definition::comparison}.
But the restriction of scalar functor commutes with truncations, so indeed $ \D (B) $ is closed under truncations.
\end{proof}

\begin{rmk}
If both $ A ^{\huflag} $ and $ B ^{\huflag} $ are static, then we can use Proposition \ref{definition::comparison} to transport the
pre-analytic animated ring strcuture on $ B ^{\huflag} $ to a pre-analytic ring structure on $ B ^{\huflag} $,
even if the functor $ -\dten _{A ^{\huflag}}B ^{\huflag} $ does not preserve the heart.
\end{rmk}

\subsection{Completing a pre-analytic ring}

The final piece of abstract nonsense is a recipe to complete a pre-analytic (animated) ring into an analytic (animated) ring.
The idea is to take the completion of $ A ^{\huflag} $, which is an object of $ \D (A) $.
However the completion can fail to be static, which makes it hard to control the $ \exts ^{i} $ if we do completion at the abelian level.
If we do it on the derived level, then it is difficult to get the strcuture of an animated ring on the completion:
the (symmetric monoidal)-ness of the completion only buys us that of an $ \mathbb{E}_{\infty} $-ring!
There is a full proof, which spans quite a few pages in \cite{rodriguez-camargo_notes_nodate},
that deals with the animated ring structure properly.
Here we restrict to the case where the completion of $ A ^{\huflag} $ is static
(but $ A ^{\huflag} $ itself does not have to be so),
which suffices in a lot of the applications.
\begin{prop}
Suppose that $ (A ^{\huflag}, \D (A)) $ is an analytic animated ring.
Let $ A ^{\huflag\huflag} $ be the object $ A ^{\huflag} \dten _{A ^{\huflag}} A $.
Assume that the object $ A ^{\huflag\huflag} $ is static, i.e. lies in $ \D (A ^{\huflag})^{\heart} $.
Then:
\begin{enumerate}
\item The object $ A ^{\huflag\huflag} $ is a condensed animated $ A ^{\huflag} $-algebra;
\item If $ M $ is an object of $ \D (A) $, then there is a unique $ A ^{\huflag\huflag} $-module structure on $ M $
whose restriction of scalar to $ A ^{\huflag} $ is $ M $;
\item Let $ \D (A)' $ be the essential image of the functor $ \D (A)\to \D (A ^{\huflag\huflag}) $ in (2).
The pair $ (A ^{\huflag\huflag}, \D (A)') $ is an analytic animated ring;
\item If $ M,N $ are objects in $ \D (A) $, then
\begin{equation*}
\rhoms _{A ^{\huflag}}(M, N) \simeq \rhoms _{A ^{\huflag\huflag}}(M, N).
\end{equation*}
\end{enumerate}
\end{prop}

\begin{proof}
Denote temporarily by $ L _{A} $ the functor $ -\dten _{A ^{\huflag}} A $.

(1)
Applying $ L _{A} $ to the isomorphism $ A ^{\huflag}\dten _{A ^{\huflag}} A ^{\huflag} \to A ^{\huflag} $
and using that $ L _{A} $ is symmetric monoidal gives a map 
$$ A ^{\huflag\huflag}\dten _{A ^{\huflag}} A ^{\huflag\huflag}\to L _{A} (A ^{\huflag\huflag} \dten _{A ^{\huflag}} A ^{\huflag\huflag})
\to A ^{\huflag\huflag}, $$
which makes $ A ^{\huflag\huflag} $ into a condensed $ \mathbb{E}_{\infty} $-ring.
But $ A ^{\huflag\huflag} $ is static, so it also has a structure of a condensed animated ring.

(2)
Let $ M $ be an object of $ \D (A) $. Then applying $ L _{A} $ to the isomorphism $ M\dten _{A ^{\huflag}} A ^{\huflag}\to M $ gives a map
$ M \dten _{A ^{\huflag}} A ^{\huflag\huflag}\to M $.
This maps makes $ M $ into an $ \mathbb{E}_{\infty} $-$ A ^{\huflag\huflag} $-module.

A map $ M\dten _{A ^{\huflag}} A ^{\huflag\huflag} \to M $ factors uniquely as the counit and its completion
$$M \dten _{A ^{\huflag}}A ^{\huflag\huflag}\to L _{A}(M \dten _{A ^{\huflag}} A ^{\huflag\huflag}) \to M .$$
This shows uniqueness.

(3)
%Let $ \D (A)' $ be the essential image of $ \D (A) $ in $ \D (A ^{\huflag\huflag}) $ under the construction in (2).
If $ M $ is an object of $ \D (A) $ and $ M' $ is the corresponding object in $ \D (A)' $,
then the restriction of scalar of $ M' $ to $ \D (A ^{\huflag}) $ is just $ M $.
While if $ M $ is an object of $ \D (A ^{\huflag\huflag}) $ whose restriction of scalar to $ \D (A ^{\huflag}) $, $ M _{0} $,
is in $ \D (A) $, then $ M $ is obtained by $ M _{0} $ from the construction of (2) by uniqueness.
Thus $ \D (A)' \subset \D (A ^{\huflag\huflag}) $ defines the induced pre-analytic structure on $ A ^{\huflag\huflag} $.
By constrution $ A ^{\huflag\huflag} $ lies in $ \D (A)' $,
so $ (A ^{\huflag\huflag}, \D (A)') $ is an analytic animated ring.

(4)
Suppose that $ M,N $ are objects in $ \D (A)' $ whose restrictions of scalars are $ M _{0}, N _{0} $.
Let $ L _{\huflag}, L _{\huflag\huflag} $ be the functors $ -\dten _{A ^{\huflag}} A, -\dten _{A ^{\huflag\huflag}} A $ respectively.
Then
\begin{align*}
\rhoms _{A ^{\huflag}}(M _{0}, N _{0})
&\simeq \rhoms _{A ^{\huflag\huflag}}(A ^{\huflag\huflag}\dten _{A ^{\huflag}}M _{0}, N)\\
&\simeq \rhoms _{A ^{\huflag\huflag}}(L _{\huflag\huflag}( A ^{\huflag\huflag} \dten _{A ^{\huflag}} M _{0} ), N)\\
&\simeq \rhoms _{A ^{\huflag\huflag}}(L _{\huflag\huflag}( A ^{\huflag\huflag} \dten _{A ^{\huflag}} A ^{\huflag} \dten _{A ^{\huflag}} M _{0} ), N)\\
&\simeq \rhoms _{A ^{\huflag\huflag}}(
L _{\huflag\huflag}( A ^{\huflag\huflag} \dten _{A ^{\huflag\huflag}} L _{\huflag}( A ^{\huflag\huflag} \dten _{A ^{\huflag}} M _{0} ) )
, N)\\
&\simeq \rhoms _{A ^{\huflag\huflag}}(L _{\huflag\huflag}(M), N)\\
&\simeq \rhoms _{A ^{\huflag\huflag}}(M, N)
\end{align*}
\end{proof}

By (4), we may abuse notation and denote the completion of $ (A ^{\huflag}, \D (A)) $ as $ (A ^{\huflag\huflag}, \D (A)) $.

\begin{rmk}
Completion of a pre-analytic animated ring into an analytic animated ring,
if established in full generality as in \cite{rodriguez-camargo_notes_nodate},
gives a left adjoint to the inclusion functor
\begin{equation*}
\{\text{analytic animated rings}\}\to \{\text{pre-analytic animated ring}\}.
\end{equation*}
\end{rmk}

\section{Easy examples}

\begin{exmp}
For any condensed ring $ A ^{\huflag} $, the pair $ (A ^{\huflag}, \modcat (A ^{\huflag})) $ is an analytic ring.
Also, for any condensed animated ring $ A ^{\huflag} $, the pair $ (A ^{\huflag}, \D (A ^{\huflag})) $ is an analytic animated ring.
We call this analytic (animated) ring structure the \emph{trivial} analytic (animated) ring strcuture.
\end{exmp}

\begin{exmp}
The pair $ \mathbb{Z}_{\solid} = (\mathbb{Z}, \mathrm{Solid}) $ is an analytic ring.
We call $ \mathbb{Z}_{\solid} $ the solid base ring.
\end{exmp}

In the solid case we have the following comparison:
\begin{prop}
The functor $ \D (\mathrm{Solid})\to \D (\mathrm{CondAb}) $ is fully faithful with essential image being
the solid complexes of condensed abelian groups.
\end{prop}

\begin{proof}
We only show (fully faithful)-ness.

A more care examination of the calculation of the solidification of $ P $ reveals that
$ P ^{L\solid}\simeq \prod _{\mathbb{N}}\mathbb{Z} $.
See \cite[\href{https://www.youtube.com/watch?v=bdQ-\_CZ5tl8\&list=PLx5f8IelFRgGmu6gmL-Kf\_Rl\_6Mm7juZO\&index=5\&t=3182s}{Video 5, 53:02}]{ihesvid}
for details.
Thus for all solid abelian group $ M $ and $ i\geq 0 $, the object
$ \exts ^{i}_{\mathbb{Z}}(\prod _{\mathbb{N}} \mathbb{Z}, M) $
is isomorphic to $ \exts ^{i}_{\mathbb{Z}}(P, M) $, which vanishes.

Suppose that $ M ^{\bullet}, N ^{\bullet} $ are two complexes of solid abelian groups.
The $ \rhom (M ^{\bullet}, N ^{\bullet}) $ in the category $ \D (\mathrm{Solid}) $ is calculated by replacing $ M ^{\bullet} $
with a K-projective complex,
while the $ \rhom $ in $ \D (\mathrm{CondAb}) $ is calculated by replacing $ N ^{\bullet} $ by with a K-injective complex.
We need to show that the two coincide.

By writing $ M ^{\bullet} $ as a colimit of its truncations to the right and $ N ^{\bullet} $ as a limit of its truncations to the left,
we may assume that $ M ^{\bullet}\in \D (\mathrm{Solid})^{\leq 0} $ and $ N ^{\bullet}\in \D (\mathrm{Solid})^{\geq 0} $.
Moreover, we may replace $ M ^{\bullet} $ with a complex $ P ^{\bullet} $ whose terms are direct sums of copies of $ \prod _{\mathbb{N}} \mathbb{Z} $.
Each term of $ P ^{\bullet} $ is then acyclic with respect to $ \rhom _{\mathbb{Z}}(-, Q) $
for any solid abelian group $ Q $ by the vanishing of higher $ \ext ^{i} $.
A spectral sequence calculation shows that $ \Hom (P ^{\bullet}, N ^{\bullet}) $ calculates $ \rhom (M ^{\bullet}, N ^{\bullet}) $.
\end{proof}

\begin{exmp}
Let $ A $ be any condensed animated ring. Then the map $ \mathbb{Z}\to A $ and the analytic ring structure
$ \mathbb{Z}_{\solid} $ induces a pre-analytic animated ring structure on $ A $.
We denote the completion of this analytic structure by $ A _{\solid} $.

If $ A $ is solid as a condensed abelain group, then completion is not neccessary.
In this case $ A _{\solid} = (A, \D (A _{\solid})) $ where $ \D (A _{\solid}) $ is the full subcategory
of $ \D (A) $ consisting of objects that are solid as condensed abelian groups.

Since the solidification of $ \mathbb{R} $ is $ 0 $, the analytic ring $ \mathbb{R}_{\solid} $ is the $ 0 $ ring.
\end{exmp}

\begin{exmp}
We give one example where derived solidification runs away to the left.
Let $ G $ be a topological monoid whose underlying space is a CW complex.
Then $ \mathbb{Z}[G] $ is a condensed ring.

The calculations about condensed cohomology can be used to show that $ \mathbb{Z}[G]^{L\solid} $
is isomorphic to the discrete object in $ \D (\mathrm{CondAb}) $ with value the singular complex of $ G $.
See
\cite[\href{https://www.youtube.com/watch?v=KKzt6C9ggWA\&list=PLx5f8IelFRgGmu6gmL-Kf\_Rl\_6Mm7juZO\&index=6\&t=1090s}{Video 6, 18:10}]{ihesvid}
for more details.

If $ G = S ^{1} $, then we have in fact that $ \mathbb{Z}[S ^{1}]^{L\solid} \simeq \mathbb{Z} \oplus \mathbb{Z}[1] $.
It is clear how to put a structure of a condensed animated ring on it.
\end{exmp}

\begin{exmp}
Let $ A\to B $ be a map of pre-analytic animated rings.
Then the map factor as $ A \to B _{0}\to B $, where $ B _{0}$ is the induced pre-analytic
structure on $ B ^{\huflag} $.

If $ B $ is an analytic animated ring, then so is $ B _{0} $.
\end{exmp}

\section{Gaseous theory}

\subsection{Gaseous modules}

The gaseous theory is designed to be the universal context for constructing the Tate elliptic curve.
It is global in the sense that it specializes to theories on both non-archimedean and archimedean rings.

Kedlaya chose to use \emph{coalescent} instead of \emph{gaseous}.
See \cite{kedlaya}, Section 10.7 for the idea behind the terminology.
(Make sure you use the latest version of his notes, or you will probably see \emph{liquid} which was a bad choice.)
I stick with gas, even if gas is not a condensed state of matter.

Suppose that $ A ^{\huflag} $ is a condensed ring.
Let $ P _{A} $ be the $ A ^{\huflag} $-module $ P \otimes _{\mathbb{Z}} A ^{\huflag} $,
where $ P = \mathbb{Z}[\mathbb{N}\cup \{\infty\}] / \mathbb{Z}[\infty] $.
Using the adjunction and the fact that $ P $ is internally projective,
we see that $ P _{A} $ is internally projective in $ \modcat (A ^{\huflag}) $.

\begin{defi}
Fix an element $ f\in A ^{\huflag} (*) $. An $ A ^{\huflag} $-module $ M $ is said to be \emph{$ f $-gaseous} if
\begin{equation*}
(\mathrm{id} - f \sigma)^{*}: \Homs _{A ^{\huflag}}(P _{A}, M) \to \Homs _{A ^{\huflag}}(P _{A}, M)
\end{equation*}
is an isomorphism.
A complex $ M\in \D (A ^{\huflag}) $ is said to be $ f $-gaseous if
\begin{equation}
(\mathrm{id}-f \sigma)^{*}: \rhoms _{A ^{\huflag}}(P _{A}, M)\to \rhoms _{A ^{\huflag}}(P _{A}, M)
\end{equation}
is an isomorphism.
\label{gaseous::def}
\end{defi}

The idea of $ f $-gaseous is that
if $ \{a _{i}\} $ is a null sequence in $ M $, then
$ \{a _{i}\} $ is summable against $ f $, i.e. $ \{ f ^{i} a _{i}\} $ is summable.


\begin{prop}
Using the notations from Definition \ref{gaseous::def}.
Let $ \mathsf{Gas}(f) $ be the category of $ f $-gaseous complexes in $ \D (A ^{\huflag}) $.
Then $ (A ^{\huflag}, \mathsf{Gas}(f)) $ is a pre-analytic animated ring.
\end{prop}

The proof is essentially a word-to-word copy of the solid case.

\begin{proof}
The object $ P _{A} $ is internally compact projective, thus $ \rhoms _{A ^{\huflag}}(P _{A}, -) $ commutes with all limits and colimits.
It is immediate that $ \mathsf{Gas}(f) $ is closed under limits and colimits.
Then Remark \ref{definition::nonsense} gives the left and right adjoints.
Since $ \hom _{A ^{\huflag}}(P _{A}, -) $ commutes with all $ H ^{i} $, a complex $ M $ is $ f $-gaseous if each $ H ^{i}(M) $ is $ f $-gaseous.

If $ M,N $ are $ A ^{\huflag} $-modules, then
\begin{align*}
\rhoms _{A ^{\huflag}}(P _{A}, \rhoms _{A ^{\huflag}}(M, N))
&\simeq \rhoms _{A ^{\huflag}}(P _{A}\dten _{A ^{\huflag}} M, N)\\
&\simeq \rhoms _{A ^{\huflag}}(M, \rhoms _{A ^{\huflag}}(P _{A}, N)).
\end{align*}
Using this isomorphism it is easy to see that if $ N $ is $ f $-gaseous,
then $ \rhoms _{A ^{\huflag}}(M, N) $ is also $ f $-gaseous.

Since being $ f $-gaseous can be checked on cohomology, $ \mathsf{Gas}(f) $ is closed under truncations.
Then the argument in the proof of Proposition \ref{definition::comparison} shows that
the left adjoint to $ \mathsf{Gas}(f)\to \D (A ^{\huflag}) $ preserves the connective parts.
\end{proof}

\subsection{Gaseous base ring}

Now we discuss the case where $ f $ is a topologically nilpotent unit.

The addition on $ \mathbb{N} $ induces a monoid structure on $ \mathbb{N}\cup \{\infty\} $
and in turn a ring strcuture on $ P $,
which we denote by $ \mathbb{Z}[\hat{q}] $.
There is a map from the discrete condensed ring $ \mathbb{Z}[q] $ to $ \mathbb{Z}[\hat{q}] $
that sends $ q $ to $ [1] $.
More formally, it is induced by the map $ \mathbb{N}\to \mathbb{N}\cup \{\infty\} $ of condensed monoids.
Let $ A _{0}^{\huflag} $ be the condensed ring $ \mathbb{Z}[\hat{q}][q ^{-1}]$,
which is more formally constructed as
\begin{equation*}
A _{0}^{\huflag} := \colim (\mathbb{Z}[\hat{q}] \xrightarrow{\times q} \mathbb{Z}[\hat{q}]\xrightarrow{\times q} \mathbb{Z}[\hat{q}]\xrightarrow{\times q} \cdots   ).
\end{equation*}

\begin{defi}
Let $ A $ be a condensed ring. An element $ f \in  A (*) $ is called \emph{topologically nilpotent}
if the map $ \mathbb{Z}[T]\to A $ sending $ n $ to $ f ^{n} $ factors through $ \mathbb{Z}[T]\to \mathbb{Z}[\hat{q}]$.
\end{defi}

The condensed ring $ A _{0}^{\huflag} $ is the initial (condensed ring equipped with a topologically nilpotent unit).

\begin{prop}
The completion of the pre-analytic animated ring
$ (A _{0}^{\huflag}, \mathsf{Gas}(q)) $
is $ (A ^{\huflag}, \mathsf{Gas}(q)) $ where $ A ^{\huflag} $ is the static condensed animated ring
\begin{equation*}
\bigcup _{k,c,n>0} \prod _{m\geq -n} (\mathbb{Z}\cap [-c(m+n) ^{k}, c (m+n)^{k}]) q ^{m}.
\end{equation*}
(The union is really a filtered colimit with injective transition maps that has a sequential cofinal set.
Since the objects in the colimit are profinite sets,
this colimit can be taken in either condensed sets or topological spaces.)
\end{prop}

See
\cite[\href{https://www.youtube.com/watch?v=krq6jCy-dhE\&list=PLx5f8IelFRgGmu6gmL-Kf\_Rl\_6Mm7juZO\&index=14\&t=1368s}{Video 14, 22:48}]{ihesvid}
for the proof.

\subsection{Gaseous real number}

\begin{defi}
The gaseous real number is the analytic ring $ (\mathbb{R}, \mathsf{Gas}(\frac{1}{2})) $.
\end{defi}

The choice of $ \frac{1}{2} $ is not important here. Any $ 0<\alpha <1 $ gives the same theory.

\begin{prop}
Let $ V $ be a topological $ \mathbb{R} $-vector space whose topology is induces by a $ p $-norm $ || $ for some $ 0<p\leq 1 $.
(A $ p $-norm always exists for some $ p $ whenever $ V $ is Hausdorff and locally bounded.)
Denote by $ \underline{V} $ the embedding of $ V $ into condensed $ \mathbb{R} $-vector spaces.
Then $ V $ is complete (i.e. a $ p $-Banach space) if and only if $ \underline{V} $ is gaseous.
\end{prop}

\begin{proof}
Suppose that $ V $ is complete. We construct explicitly the inverse to $ (\mathrm{id} - \frac{1}{2}\sigma)^{*} $.
For any profinite set $ S $, the sections $ \Homs _{\mathbb{R}}(P _{\mathbb{R}}, \underline{V})(S) $
can be identified with sequences of the form $ \{a _{i}\} $ where each $ a _{i}: S\to V $ is a continuous map
and the sequence $ \{a _{i}\} $ converges uniformly to $ 0 $.
Let $ b _{n} $ be the partial sum $ \sum _{j\leq n} \frac{1}{2 ^{j}} a _{j} $.
Then we have
\begin{align*}
\sup _{S}|b _{n+k} - b _{n}|
&= \sup _{S}|\sum _{j=n+1}^{n+k} \frac{1}{2 ^{j}}a _{j}|\\
&\leq \sum _{j=n+1}^{n+k} \frac{1}{2 ^{jp}} \sup _{S}|a _{j}|
\end{align*}
which tends to $ 0 $ since $ \{a _{n}\} $ uniformly converges to $ 0 $.
Therefore for each point $ x\in S $, the sequence $ \{b _{n}(x)\} $ is Cauchy, so converges to some limit $ b (x) $.
The estimation also shows that $ \{b _{n}\} $ converges uniformly to $ b (x) $,
which implies that $ x\mapsto b (x) $ is continuous.
Thus $ \{a _{i}\} $ is summable against $ \frac{1}{2} $.
The summation gives an inverse to $ (\mathrm{id}-\frac{1}{2}\sigma)^{*}(S) $.
As the sums are neccessarily unique, the inverses for various $ S $ are compatible,
and give the desired inverse map.

Conversely, suppose that $ \underline{V} $ is gaseous.
Let $ \{a _{i}\} $ be a Cauchy sequence in $ V $.
Passing to a subsequence, we may assume that $ |a _{i} - a _{i+1}| < 2 ^{-i(1+p)} $ for each $ i $.
Then $ \{2 ^{i}(a _{i+1}- a _{i})\} $ is a null sequence in $ V $, which corresponds to an element of
$ \Homs _{\mathbb{R}}(P _{\mathbb{R}}, \underline{V})(*) $.
As $ \underline{V} $ gaseous, the sequence $ \{2 ^{i}(a _{i+1}-a _{i})\} $ is summable against $ \frac{1}{2} $,
which gives a limit of $ \{a _{i}\} $.
\end{proof}

So the gaseous real number captures the completeness for Hausdorff, locally bounded topological vector spaces.
Note that the proof used the scaling behaviour of the $ p $-norm in an essential way,
and thus does not work without the locally bounded assumption.
\appendix
\section{Condensed Anima}

The notion of condensed animated sets/abelian groups/rings behave very well in the old condensed framework of \cite{condensed},
as the old category of condensed sets/abelian groups/rings are generated by compact projective objects of the form $ \mathbb{Z}[S] $
where $ S $ is extremally disconnected.
However, we lose most of these projective objects in the light setting.
Thus we need to put more care into defining condensed derived objects.

We start with the $ \infty $-topos where all condensed derived objects live.
\begin{defi}
A \emph{condensed anima} is a hypersheaf of anima on the site $ \mathrm{Pro}_{\mathbb{N}}(\mathsf{FinSet}) $.
The $ \infty $-category of condensed anima is denoted by $\condani$.
\end{defi}

A sequential limit of coverings in the site $ \mathrm{Pro}_{\mathbb{N}}(\mathsf{FinSet}) $ is also a covering.
Thus the topos of condensed sets is replete.
A result of Mondal and Reinecke in \cite{mondal_postnikov_2024} shows that $ \condani $ is Postnikov complete.

\begin{defi}
If \widget{} is an algebraic structure, then
a \emph{condensed animated \widget{}} is a \widget -object in $ \condani $.
The $ \infty $-category of condensed animated \widget{} is denoted by $ \condani\widget $.
\end{defi}

If \widget{} is abelian groups, there is another reasonable way to define condensed animated abelian groups:
the category $ \mathsf{CondAb} $ is a Grothendieck abelian category and we can construct its (unbounded) derived $ \infty $-category
using the injective model structure on the category of chain complexes.
Fortunately the two approaches agree.
\begin{prop}
There is an equivalence $ \D (\mathsf{CondAb})\to \mathrm{HypSh}(\mathrm{Pro}_{\mathbb{N}}(\mathsf{FinSet}), \D (\mathbb{Z})) $.
\end{prop}

\begin{proof}
There is a $ t $-structure on the right hand side
where an object $ X $ is connective if and only if all sections $ X (U) $ on light profinite sets are connective.
The heart of this $ t $-structure is $ \mathsf{CondAb} $.
Then we can
apply \cite{sag}, Theorem 2.1.2.2 to the $ \infty $-topos $ \condani $ with $ \mathcal{O} = \mathbb{Z} $.
\end{proof}

Note that the proof relies on the hypercomplete-ness of $ \mathsf{CondAni} $.



\printbibliography
\end{document}

