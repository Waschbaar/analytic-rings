\section{Condensed Anima}

Warning: it is important to use the derived $ \infty $-category instead of the ordinary derived category in this section.

The notion of condensed animated sets/abelian groups/rings behave very well in the old condensed framework of \cite{condensed},
as the old category of condensed sets/abelian groups/rings are generated by compact projective objects.
However, we lose most of these projective objects in the light setting.
In fact, there are \emph{not} enough projectives in the category of light condensed abelian groups.
Thus we need to put more care into defining condensed derived objects.

We start with the $ \infty $-topos where all condensed derived objects live.
\begin{defi}
A \emph{condensed anima} is a hypersheaf of anima on the site $ \mathrm{Pro}_{\mathbb{N}}(\mathsf{FinSet}) $.
The $ \infty $-category of condensed anima is denoted by $\condani$.
\end{defi}

A sequential limit of coverings in the site $ \mathrm{Pro}_{\mathbb{N}}(\mathsf{FinSet}) $ is also a covering.
Thus the topos of condensed sets is replete.
A result of Mondal and Reinecke in \cite{mondal_postnikov_2024} shows that $ \condani $ is Postnikov complete.

\begin{defi}
If \widget{} is an algebraic structure, then
a \emph{condensed animated \widget{}} is a \widget -object in $ \condani $.
The $ \infty $-category of condensed animated \widget{} is denoted by $ \condani\widget $.
\end{defi}

If \widget{} is abelian groups, there is another reasonable way to define condensed animated abelian groups:
the category $ \mathsf{CondAb} $ is a Grothendieck abelian category and we can construct its (unbounded) derived $ \infty $-category
using the injective model structure on the category of chain complexes.
Fortunately the two approaches agree.
\begin{prop}
There is an equivalence $ \D (\mathsf{CondAb})\to \condani \mathsf{Ab} $.
\end{prop}

\begin{proof}
There is a $ t $-structure on $ \condani $ where an object $ X $ is connective if and only if all sections $ X (U) $ on light profinite sets are connective.
The heart of this $ t $-structure is $ \mathsf{CondAb} $.
Then we can
apply \cite{sag}, Theorem 2.1.2.2 to the $ \infty $-topos $ \condani $ with $ \mathcal{O} = \mathbb{Z} $.
\end{proof}

Note that the proof relies on the hypercomplete-ness of $ \mathsf{CondAni} $.

The notion of condensed anima allows mixing two perspectives of topological spaces,
in the sense that we have the following two functors from $ \mathsf{Top} $ to $ \condani $:
\begin{itemize}
\item the \emph{body} functor $ \material: \mathsf{Top}\to \mathsf{CondSet}\to \condani $ where the first functor sends a topological space $ X $
to the condensed set $ \underline{X} $ and the second functor is viewing a condensed set as a static condensed anima;
\item the \emph{spirit} functor $ \spirit: \mathsf{Top}\to \mathsf{Ani}\to \condani $ where the first functor takes a topological space $ X $
to its fundamental $ \infty $-groupoid $ |X| $ and the second functor is the constant sheaf construction.

Lei $ \mathsf{mLCA} $ be the category of metrizable locally compact abelian groups.
\end{itemize}

